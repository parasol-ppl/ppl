\documentclass[12pt]{article}

\usepackage{algorithmic}
\usepackage{fullpage}
\usepackage{parskip}
\usepackage{url}

\usepackage[usenames]{color}
\usepackage{listings}
\definecolor{LightGrey}{rgb}{0.9,0.9,1.0}
\definecolor{gray}{rgb}{0.4,0.4,0.4}
\definecolor{darkblue}{rgb}{0.0,0.0,0.6}
\definecolor{cyan}{rgb}{0.0,0.6,0.6}
\lstdefinestyle{C++}{language=C++,%
  identifierstyle=\color{black},
  backgroundcolor=\color{LightGrey},
  showstringspaces=false,
  columns=fullflexible,
  basicstyle=\sffamily,
  commentstyle=\color{gray},
  moredelim=**[is][\color{white}]{~}{~},
  literate={=>}{{$\Rightarrow\;$}}1 {->}{{$\rightarrow{}$}}1 %
  {<-}{{$\leftarrow$}}1 {<:}{{$\subtype{}\ $}}1 {*}{{$^*$}}1 %
  {\{}{{$\{$}}1 {\}}{{$\}$}}1 {<}{{$<$}}1 {>}{{$>$}}1,
}
\lstdefinelanguage{XML}{
  morestring=[b]",
  morestring=[s]{>}{<},
  morecomment=[s]{<?}{?>},
  identifierstyle=\color{darkblue},
}
\lstdefinestyle{XML}{
  language=XML,
  backgroundcolor=\color{LightGrey},
  showstringspaces=false,
  columns=fullflexible,
  basicstyle=\ttfamily,
}

\newcommand{\pmpl}{PMPL}
\newcommand{\gforge}{\texttt{GForge}}

\title{\Large \bf Standards and Procedures of \pmpl}
\author{Jory Denny and Read Sandstr\"{o}m and Shawna Thomas and Nancy M. Amato}

\begin{document}

\lstset{style=C++}

\maketitle
\thispagestyle{empty}
\pagestyle{empty}

This document outlines the standards for all procedures within \pmpl. This
ranges from coding standards, to testing standards, to documentation standards,
etc. Everyone will be expected to adhere to these standards.

\clearpage
\pagestyle{plain}

%%%%%%%%%%%%%%%%%%%%%%%%%%%%%%%%%%%%%%%%%%%%%%%%%%%%%%%%%%%%%%%%%%%%%%%%%%%%%%%%
%% Check-in procedure
%%%%%%%%%%%%%%%%%%%%%%%%%%%%%%%%%%%%%%%%%%%%%%%%%%%%%%%%%%%%%%%%%%%%%%%%%%%%%%%%

\section{Check-in Procedure}
The following procedure will occur anytime a check-in will take place within
\pmpl. No warnings or errors are permitted in check-ins.
\begin{algorithmic}[1]
\STATE The check-in should be documented in a task on \gforge\ (for\
non-simplistic check-ins).
\STATE One person will check over correctness and testing for proposed code.
This person should be an experienced group member in the component of \pmpl\
related to the check-in. The discussion of correctness should be documented on
\gforge.
\STATE A separate person shall check over the coding and documentation standards
set forth within this document. This person can be any member of the motion
planning group and should be as nitpicky as possible. Any discussion of fixes in
standards should be documented on \gforge.
\STATE If the check-in involves a new algorithm, the documentation for \pmpl\
must be augmented to include this new algorithm.
\STATE Once approval is made on correctness and standards, the check-in can be
made and \gforge\ task closed.
\STATE Back up validation shall occur during the \pmpl\ nightly tests, person
who checks in the code is responsible for fixing warnings and errors caused by
the check-in.
\end{algorithmic}

%%%%%%%%%%%%%%%%%%%%%%%%%%%%%%%%%%%%%%%%%%%%%%%%%%%%%%%%%%%%%%%%%%%%%%%%%%%%%%%%
%% Coding Standards
%%%%%%%%%%%%%%%%%%%%%%%%%%%%%%%%%%%%%%%%%%%%%%%%%%%%%%%%%%%%%%%%%%%%%%%%%%%%%%%%

\section{Coding Standards}

\subsection{\texttt{.h/.cpp}}
All classes and files will be split between \texttt{.h/.cpp} format when
possible, i.e., non-template classes/functions. All \texttt{.h} files should
have \#define guards with the name of the file in all caps, with \texttt{\_}s
separating words and \texttt{\_H\_} at the end. For example: for
\texttt{ExampleClass.h} -- \texttt{PMPL\_EXAMPLE\_CLASS\_H\_}.

\begin{lstlisting}
#ifndef PMPL_EXAMPLE_CLASS_H_
#define PMPL_EXAMPLE_CLASS_H_

class Class;

#endif
\end{lstlisting}

Newer files will include `\texttt{PMPL\_}' at the front of the include guard to
further reduce the possibility of name collision with an external library.
Please use this prefix in all new include guards.

\subsection{Naming}

\subsubsection{\#define}
All \#define preprocessor commands should be named in all capital letters, with
\_s separating words.

\begin{lstlisting}
#define MY_DEFINE
\end{lstlisting}

When possible, \#defines should be avoided and use constant functions instead.
This is the C++ style.

\subsubsection{Types}
All types, being all classes, structs, typedefs, enums, etc, should begin
with a capital letter and be in camel casing format (i.e., each successive word
will be capitalized).

\begin{lstlisting}[mathescape]
class MyClass;
struct MyStruct;
typedef int MyInt;
enum MyEnum {Value1, Value2, Value3};
\end{lstlisting}

\paragraph{Pointer and Reference Type Specification}

When declaring a variable to be a pointer or reference the * or \& should be
adjacent to the type:

\begin{lstlisting}
Type* pointerName;
Type& referenceName;
\end{lstlisting}

\subsubsection{Enums}
Enum values should be named like types, i.e, camel casing beginning with a
capital letter.

\begin{lstlisting}[mathescape]
enum MyEnum {Value1, Value2, Value3};
\end{lstlisting}

\subsubsection{Functions}
Functions shall be names with camel casing beginning with a capital letter.

\begin{lstlisting}
void MyFunction();
\end{lstlisting}

\subsubsection{Variables}

All variables should be named in camel casing beginning with a lowercase letter.

\begin{lstlisting}
int myInt;
\end{lstlisting}

\paragraph{Member/class variables}

All member variables should begin with an m\_. No type specification is
necessary (i.e., no m\_pVariable for a pointer).


\begin{lstlisting}[mathescape]
class MyClass {
  public:
    int m_myInt;
}
\end{lstlisting}

\paragraph{Function parameters}

All function parameters should begin with an \_.

\begin{lstlisting}
int MyFunction(int _a, int _b);
\end{lstlisting}

\subsubsection{Naming Exceptions}

The following cases are exempt from naming conventions for language specific
reasons.

\begin{itemize}

  \item Functions defining iterators, namely, \texttt{begin()}, \texttt{end()},
    and the iterator typedef. This is for compatability with \texttt{auto}
    keyword for range based for loops.

  \item In classes which extend part of the STL, the methods which override an
    STL class's member function must be named against our standard, obviously.

\end{itemize}

\subsection{XML}
Similarly to the class and variable names, within the XML, tags should be camel
cased beginning with a capital letter and variables should be camel cased
beginning with a lowercase letter.

\lstset{style=XML}
\begin{lstlisting}
<MyXMLTag label="label" myXMLVariable="false"/>
<MyXMLTagWithChildren label="label2">
  <Child method="label"/>
  <Child method="label2"/>
</MyXMLTagWithChildren>
\end{lstlisting}
\lstset{style=C++}

When loading labels for \pmpl\ methods, e.g., distance metrics, the variable
name should be the \pmpl\ acronym for that component with Label. So for a
distance metric label variable it should be named dmLabel, a validity checker
label should be vcLabel, etc. Acronyms are as follows:

\begin{itemize}
  \item Distance Metric - dm
  \item Validity Checker - vc
  \item NeighborhoodFinder - nf
  \item Sampler - s
  \item Local Planner - lp
  \item Connector - c
  \item Map Evaluator - me
  \item Metric - m
  \item Extender - e
  \item Path Modifier - pm
  \item Motion Planning Strategy - mps
\end{itemize}

\subsection{Spaces/Tabbing}
All tab characters should be represented by 2 spaces (easy to make default in
all editors, vim is recommended). There should be no trailing white space at the
end of lines (this causes unnecessary conflicts with in SVN).

\subsection{Line length}
All lines of code should be a reasonable length for code clarity. 80 characters
is an expected line limit. Though this is not a strictly enforced standard,
please try to adhere as best as possible to accomodate those working on smaller
screens.

\subsection{Curly Braces}
Curly braces used in control flow, function definitions, or class definitions
should be placed on the same line as the statement. It is recommended that a
space occurs before the curly brace, e.g., if() \{ NOT if()\{.

\begin{lstlisting}
class MyClass {
  void MyFunction() {
    while(1) {
    }
  }
}
\end{lstlisting}

Additionally, if the body of control flow structures is one line, curly braces
are not required, and the body must appear on a new line.

\begin{lstlisting}
if(true)
  DoSomething();
\end{lstlisting}

\subsection{Function Call/Declaration/Definition}
Function calls or declarations which are "long" should be split between multiple
lines when necessary to aid in reading the code.

Definitions of member functions should be outside the class definition, even for
one-line functions (this helps fully separate a class's implementation from its
interface). The exception here is for small classes which serve as containers or
functors: in these cases, member definitions may be placed directly within the
class.

Definitions shall be formatted with the template, return type, class name, and
function name all on separate lines. The function parameters may span multiple
lines if necessary.

\begin{lstlisting}
template<class MPTraits>
ReturnType
ClassName<MPTraits>::
FunctionName(parameter list which may span multiple lines) {
}
\end{lstlisting}

\subsection{Overriding Functions}

All overriding functions must be declared \texttt{virtual} and
\texttt{override}. This clarifies that the function does override some base
implementation and allows the compiler to throw an error if we inadvertently
fail to match the base function's prototype.

\subsection{MPBaseObject inherited variables/functions}

\subsubsection{m\_debug}

There will be a variable, m\_debug in the PMPL base class which will be used to
determine when output occurs, e.g., in debugging statements. This means that
there will be no \#Define statements for debug output. All debug output will be
of the form:

\begin{lstlisting}
if(m_debug)
  cout << ``My debug statement'';
\end{lstlisting}

\subsubsection{GetNameAndLabel()}

This function returns a unique identifier to this class based on XML input.
Returns m\_name + ``::'' + m\_label. This should be used in output and record
keeping.

\begin{lstlisting}
string callee = this->GetNameAndLabel();
\end{lstlisting}

\subsubsection{Print}

In all classes one function Print should be implemented which displays the
values of the class variables at the time of calling. This is called for every
loaded variable after parsing the XML in the main function of PMPL.

\begin{lstlisting}
void Print(ostream& _os) const {
  _os << GetNameAndLabel() << endl;
  //output rest of class values
}
\end{lstlisting}

\subsubsection{XML Parsing}

All derived classes from MPBaseObject may support constructors that take an
\texttt{XMLNode} parameter. In all such cases, the \texttt{XMLNode} parameter
must be passed up the class hierarchy so that parent classes get an opportunity
to parse their parameters.

\subsection{Throwing Errors}

All errors, unless otherwise approved should be thrown using a PMPLException, or
other base type. This is to encourage recovering from errors in sister programs,
like Vizmo and GroupBehaviors. Exception classes are located within
\texttt{src/Utilities/PMPLExceptions.h}. Use the WHERE define to give an exact
location of the error. The PMPLException base is nicely formatted.

\begin{lstlisting}
throw PMPLException(``MyExceptionType'', WHERE, ``My error message'');
\end{lstlisting}

Our exceptions also support ostream-like usage to help with printing non-string
data that support \texttt{operator<<}, such as \texttt{Boundary} objects. This
example shows a \texttt{RunTimeException} for a boundary object:

\begin{lstlisting}
Boundary b;
throw RunTimeException(WHERE) << "Something is wrong with this boundary:\n"
    << b << std::endl;
\end{lstlisting}

\subsection{STL and External Libraries}

The STL and External Libraries, e.g, boost and CGAL should be used as much as
possible, except when there is a good reason not to, e.g., efficiency.

\subsection{Statistics Tracking}

All statistics should be tracked and recorded in the StatClass. There are
standard map data structures to associate names with statistics, e.g., sampling
attempts. The name should include the class name and label. There is a separate
map for each piece of the code and the output of such statistics is standardized
to aid in automated data collection.

\subsection{Example Class}
Example class is located in \texttt{ExampleClass.h} within this same folder:

\lstinputlisting{ExampleClass.h}

%%%%%%%%%%%%%%%%%%%%%%%%%%%%%%%%%%%%%%%%%%%%%%%%%%%%%%%%%%%%%%%%%%%%%%%%%%%%%%%%
%% Documentation Standards
%%%%%%%%%%%%%%%%%%%%%%%%%%%%%%%%%%%%%%%%%%%%%%%%%%%%%%%%%%%%%%%%%%%%%%%%%%%%%%%%

\section{Documentation}
All documentation should be done with Doxygen. This style was chosen to make
documentation very obvious in the code, and to support automatic generation of
web-viewable docs for PMPL. Note that the blank comment lines shown in these
templates are required by Doxygen in order to introduce paragraphs.

The Doxygen documentation will be compiled nightly and copied onto
Nancy's group's intranet page on here:

\url{https://parasol-webdev.cs.tamu.edu/groups/amatogroup/intranet/PMPLDocs}.

\subsection{General Documentation Proceedures}

\begin{enumerate}
  \item The function operator of a function object (e.g., std::plus) doesn’t
    need to be documented. The documentation of the class will provide all
    necessary information.
  \item Constructors, destructors, and assignment operators don’t have to be
    documented, unless they do more than simple initialization and cleanup of
    the data members of the class.
  \item Template parameters do not need to be documented so long as their name
    unambiguously describes their semantic meaning. I.e., template parameter
    names like `FloatingPointType' and `CfgType' are self-documenting in terms
    of what type is expected.
\end{enumerate}

\subsection{Class Documentation}

Place this comment block before the class definition. Document user-visible
defects and restrictions of the class in the optional section provided. Note
that the top/bottom commenting lines should extend to the 80th character across.

\begin{lstlisting}[mathescape]
///////////////////////////////////////////////////////////////////////////////
/// What information does this class abstract?
///
/// [details about the purpose of the class, for example:]
/// [If the class is implemented from a specific paper, cite authors, conference, and year]
/// [Non-obvious reasons for its place in the inheritance hierarchy, etc.]
///
/// [optional details about user-visible defects and restrictions]
/// $@$bug What is the problem
/// $@$remarks What test shows the problem?
/// $@$todo What must be done before the problem can be fixed?
///
/// $@$bug another bug
/// ...
///
/// $@$ingroup What group of classes does this belong to?
///////////////////////////////////////////////////////////////////////////////
\end{lstlisting}

\subsection{Enumeration Types}

Enums and enum classes should also be documented with a doxygen block like
classes. This should explain the meaning of the enumeration. The allowed values
should have names which clearly describe their meaning, and may additionally be
documented with in-line doxygen comments. For example, this enumeration type
describes early termination conditions for a single-source shortest path search:

\begin{lstlisting}[mathescape]
////////////////////////////////////////////////////////////////////////////////
/// The possible early-termination conditions for an SSSP run.
////////////////////////////////////////////////////////////////////////////////
enum class SSSPTermination {
  Continue,   ///< Proceed as usual.
  EndBranch,  ///< End the branch at this vertex and do not relax its neighbors.
  EndSearch   ///< End the entire search process.
};
\end{lstlisting}

\subsection{Function Documentation}

Place this comment block before the method definition.

It is not necessary to include self-evident information in the comment blocks.
In particular, there is no need to put ”default constructor” or ”destructor” in
the brief description, since that is evident from the name of the method.
Doxygen comments should not include information on the implementation of the
method.

\begin{lstlisting}[mathescape]
/// What does this method do?
/// $@$param name meaning of this parameter?
/// $@$param ...
/// $@$return What is the meaning of the return value?
///
/// Optional: if the usage is not obvious, a 'usage' section can be helpful.
/// $@$usage
/// $@$code
/// int i = 0;
/// MyFunc(i);
/// $@$endcode
///
/// Optional: if this is a non-member function and it belongs to some other semantic group.
/// $@$ingroup Semantic Group Name
\end{lstlisting}

For void functions with no parameters the following shorthand can be used (as
long as it fits nicely on one line):
\begin{lstlisting}[mathescape]
void Foo(); ///< My brief description
\end{lstlisting}

For functions with obvious return values or parameter interpretations (such as
getters/setters), the \texttt{@param} and \texttt{@return} documentation can be
omitted. For these types of functions, this usually just repeats what is stated
in the brief description.

\subsubsection{Member Function Groups}

This is optional, but if you would like to group members in your class together
based upon common functionality, this is acceptable and encouraged. An example,
would be grouping constructors, different types of computations, or different
sections of accessors. If this is desired, the following standard is enforced:

\begin{lstlisting}[mathescape]
///$@$name My Group Name
///$@${

void Foo1();
void Foo2();
$\ldots$
void FooN();

///$@$}
\end{lstlisting}

Also, grouping functions for common documentation is accepted, and is noted like
this:

\begin{lstlisting}[mathescape]
///$@${
/// Common documentation

void Foo1();
void Foo2();

///$@$}
\end{lstlisting}

\subsubsection{Overloaded Functions}

For overloaded functions. Put the long/descriptive comment at the most general
function, then for the others put the following, with the same brief as the
other overloaded functions and a short description of the differences.

\begin{lstlisting}[mathescape]
/// The long brief which appears over each overload.
/// (params go here)
/// $@$overload
/// The specific differences for this version of the function.
\end{lstlisting}

\subsection{Overridden Functions}

Functions which override a base class implementation do not need to be
documented; Doxygen will use the base class documentation for these.

\subsubsection{Inherited/derived Functions}
For inherited/derived functions, if the comment block is in the parent class,
no further documentation is needed (Doxygen uses parent classes comment by
default). Simply describe the differences/purpose of that function in the class
comment block. For example, for $\mathtt{TranslateAtS}$ local planner: no
commenting of the $\mathtt{IsConnected}$ function is necessary simply state the
description of the algorithm in the class block.

\subsection{Variable Documentation}

All class (member) and public (global) variables should be documented with their
purpose.

For class variables:
\begin{lstlisting}
bool m_myVar; ///< Brief description after the member variable.

/// This variable has a very long description which doesn't fit after the declaration.
std::vector<double> m_myVarWithLongDescription;
\end{lstlisting}

For global variables:
\begin{lstlisting}[mathescape]
/// Brief description of variable
/// $@$ingroup Which group does this variable belong (optional)?
\end{lstlisting}

The best documentation of temporary variables is in how they are named. If
additional documentation for temporaries is required, this should not use the
doxygen format (use a regular comment instead).

\subsection{Algorithm Documentation}

Any other documentation is up to the discretion of the programmer, but C++ style
comments should be used. This should be for commenting algorithms and
non-obvious code sections.

In general, comments within function implementations should be non-doxygen type
and should refer to \emph{what} the code is doing rather than \emph{how} it is
being done. The code itself is the best descriptor of the \emph{how}; comments
can help other programmers quickly understand the meaning of the code.

%%%%%%%%%%%%%%%%%%%%%%%%%%%%%%%%%%%%%%%%%%%%%%%%%%%%%%%%%%%%%%%%%%%%%%%%%%%%%%%%
%% Testing Standards
%%%%%%%%%%%%%%%%%%%%%%%%%%%%%%%%%%%%%%%%%%%%%%%%%%%%%%%%%%%%%%%%%%%%%%%%%%%%%%%%

\section{Testing}
Currently only nightly compilation testing occurs nightly. This should be
expanded in the future. However, there should be one up-to-date (working)
example of each method in the \texttt{Examples/CfgExamples.xml} file.

\end{document}
